\chapter{Descripción de la Empresa y el Sistema}

\section{Descripción de la Empresa}

\subsection{Historia}

% Modelo de negocios?

% Estructura Organizacional?

\subsection{Áreas principales de la empresa}

\section{Descripción del Sistema (SISTAL)}

\section{Objetivos}

\subsection{Objetivo General}

Diseñar la arquitectura completa del sistema de gestión de uniformes corporativos basado en microservicios en la nube, que sustituya al sistema actual y garantice mayor eficiencia operacional, escalabilidad y mantenibilidad.

% Se deberá poner esto aquí???
Como parte del proyecto, se desarrollará únicamente un módulo del sistema, a modo de implementación inicial y validación de la propuesta.


\subsection{Objetivos Específicos}

\begin{itemize}
    \item Analizar la arquitectura actual del sistema monolítico de gestión de uniformes, identificando limitaciones y oportunidades de mejora en escalabilidad y mantenibilidad.
    \item Analizar y documentar los flujos de datos a nivel de negocio con el propósito de modelarlos y replicarlos en la nueva arquitectura del sistema, asegurando su correcta adaptación y continuidad operativa en el proceso de migración.
    \item Diseñar una nueva arquitectura de microservicios escalable en la nube que permita la gestión eficiente del proceso de inventarios, pedidos, distribución y seguimiento de uniformes corporativos, considerando patrones de diseño modernos y mejores prácticas de desarrollo.
    \item Desarrollar un módulo del sistema, a modo de implementación inicial y validación de la propuesta.
\end{itemize}


\section{Alcances y Limitaciones}

\subsection{Alcances}

\begin{itemize}
    \item Reestructuración del sistema monolítico de SISTAL hacia una arquitectura de microservicios.
    \item Diseño de microservicios core para gestión de usuarios, inventarios, pedidos y estados.
    \item Configuración de infraestructura en la nube (GCP).
    \item Implementación de integración y despliegue continuo (CI/CD).
    \item Desarrollo e implementación de módulo de administración del sistema.
\end{itemize}

\subsection{Limitaciones}

\begin{itemize}
    \item No se contempla el desarrollo de aplicaciones móviles nativas o complementarias.
    \item No se usarán datos reales.
    \item No habrá capacitación masiva a usuarios finales.
    \item No se implementarán integraciones con sistemas externos no contemplados en la versión original.
    \item Se desarrollará únicamente el módulo de administración.
\end{itemize}


