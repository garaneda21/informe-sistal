% Formato de la hoja
\documentclass[letterpaper, 12pt, oneside]{book} % carta, letra: 12
\renewcommand{\baselinestretch}{1.5} % Interlineado general
\usepackage[margin=2.5cm, top=4cm, left=4cm]{geometry}

% Fuente 
%\usepackage{fontspec}
%\setmainfont{Arial}
\usepackage{helvet}
\renewcommand{\familydefault}{\sfdefault}
\usepackage[T1]{fontenc}

% Para insertar Imágenes
\usepackage{graphicx}
\graphicspath{{./figuras/}}

% Paquetes
\usepackage[english, spanish, es-tabla]{babel} % configuración en español
\usepackage{wrapfig} % envolver imágenes alrededor de texto
\usepackage{setspace} % opciones de espaciado
\usepackage{parskip} % formato de texto con salto de párrafo
\usepackage{tabularx} % mejores tablas
\usepackage[dvipsnames, table]{xcolor} % definir colores
\usepackage{hyperref} % insertar enlaces dentro del documento
\usepackage{xurl} % insertar url's directamente
\usepackage{verbatim} % para usar texto monoespaciado como código
\usepackage{listings} % para formatear texto como código
\usepackage{xcolor} % definir colores
\usepackage{enumitem} % para configurar los entornos enumerate, itemize.
\usepackage[titletoc]{appendix} % definir el apéndice
\usepackage{caption}
\usepackage{fancyhdr}
\usepackage{lastpage}

\definecolor{light-gray}{gray}{0.95}

% Colores de Enlaces (Links)
\hypersetup{
    colorlinks,
    citecolor=black,
    filecolor=black,
    linkcolor=black,
    urlcolor=blue
}

% Configuración del estilo de código
\lstset{
    basicstyle=\ttfamily\footnotesize,
    keywordstyle=\color{blue}\bfseries,
    commentstyle=\color{gray}\itshape,
    stringstyle=\color{teal},
    backgroundcolor=\color{light-gray},
    numbers=left,
    numberstyle=\scriptsize,
    stepnumber=1,
    numbersep=8pt,
    breaklines=true,
    tabsize=4,
    showstringspaces=false
    aboveskip=5pt,
    belowskip=5pt,
    lineskip=-1pt
}

% Modificar Headers y Footers
\pagestyle{fancy}
\fancyhf{}
\renewcommand{\headrulewidth}{0pt} % quitar línea superior
\fancyfoot[R]{\thepage}
\fancypagestyle{plain}{}

% Para la bibliografía
\usepackage[backend=biber, style=apa, sorting=nyt]{biblatex}
\usepackage{csquotes}
\addbibresource{referencias.bib}

%%% COMANDOS PERSONALIZADOS %%% 

% \sourcefig{nombre}{año}{url} : para insertar una fuente a una figura
\newcommand{\sourcefig}[3]{
    \caption*{\footnotesize \textbf{Fuente: #1, #2, Enlace:} \url{#3}}%
}
% \code{} : para insertar una palabra de texto monoespaciada, como documentación
\newcommand{\code}[1]{\colorbox{light-gray}{\texttt{#1}}}

%%%%%%%%%%%%%%%%%%%%%%%%%%%%%%%%%%%%%%%%%%%%%%%%%%%%%%%%%%%%%%%%%%%%
% INICIO DOCUMENTO %%%%%%%%%%%%%%%%%%%%%%%%%%%%%%%%%%%%%%%%%%%%%%%%%
%%%%%%%%%%%%%%%%%%%%%%%%%%%%%%%%%%%%%%%%%%%%%%%%%%%%%%%%%%%%%%%%%%%%

% PAGINAS PRINCIPALES
    % Derecho de Autor
    % Calificación
    % Dedicatoria (optativo)
    % Tabla de Contenido
    % Índice de Tablas
    % Índice de Ilustraciones
    % Resumen
    % Abstract
% TEXTO
    % Cuerpo de Obra
    % Citas
    % Notas
    % Conclusiones
% PÁGINAS FINALES
    % Bibliografía
    % Glosario ?????
    % Anexos

% DIAGRAMACIÓN DEL TEXTO
%   Para diagramar el texto se deben seguir las siguientes normas:
%   - Inicio de cada capítulo en una nueva página.
%   - Inicio de los títulos en la parte superior de la página, en margen centrado, letras mayúsculas y en negrita.
%   - Inicio del texto después de dos espacios bajo el título.
%   - Inicio de cada párrafo con una sangría de 5 espacios.
%   - Un espacio entre párrafos.
%   - Texto sólo por un lado de la hoja.
%   - Utilización de reglas gramaticales para separación de sílabas.
%   - Uso de mayúsculas en los títulos de las páginas preliminares.
%   - Uso de minúsculas en subtítulos, con excepción de la primera letra de la primera palabra y
%   - sustantivos propios

% ESPACIAMIENTO
% El texto de la tesis se escribe en espacio de 1,5 líneas, exceptuando los siguientes casos:
% - Textos a espacio simple: notas y citas textuales; párrafos de listas; índices de ilustraciones y tablas; y anexos.
% - Textos a espacio doble: bibliografía.
    
\begin{document}

    % ================ Preliminares ====================
    \frontmatter % números romanos

    \begin{titlepage}
    \begin{wrapfigure}[5]{L}{2cm}
        \includegraphics[width=2cm]{figuras/logo_utem.png}
    \end{wrapfigure}

    {\setstretch{1.1}
        \noindent UNIVERSIDAD TECNOLÓGICA METROPOLITANA\\
        FACULTAD DE INGENIERÍA\\
        DEPARTAMENTO DE INFORMÁTICA\\
        ESCUELA DE INFORMÁTICA\\
        CARRERA INGENIERÍA EN INFORMÁTICA
        \vfill
    }

    % TÍTULO
    \begin{center}
        \setstretch{2} % título a doble espacio si utiliza más de un reglón
        \textbf{\Large REINGENIERÍA DE UN SISTEMA DE GESTIÓN DE UNIFORMES CORPORATIVOS} 
    \end{center}

    \vfill
    
    \begin{center}
        \textbf{\large TRABAJO DE TITULACIÓN PARA OPTAR AL TÍTULO DE INGENIERO EN INFORMÁTICA} 
    \end{center}
        
    \vfill

    \begin{flushright}
        \textbf{\large AUTORES:} 

        ARANEDA MELLA, GERARDO GABRIEL \\
        GÓMEZ RODRÍGUEZ, ANDRÉS \\
        JIMÉNEZ ROMERO, NICOLÁS 
        
        \textbf{\large PROFESOR GUÍA:}
        
        YELKA RUIZ ROCHA
        
    \end{flushright}
    
    \vspace{0.5in}

    \begin{center}
        \small SANTIAGO - CHILE \\
        2025
    \end{center}

\end{titlepage}

    \setcounter{page}{2}

    \begin{singlespace}
    \begin{center}
        \large \textbf{Autorización para la Reproducción del Trabajo de Titulación}
    \end{center}
    \vspace{1cm}
    
    1. Identificación del trabajo de titulación
    
    Nombre del(os) alumno(s)
    
    \dotfill \\
    Rut \dotfill \\
    Dirección \dotfill \\
    E-mail \dotfill \\
    Teléfono \dotfill

    Título de la tesis

    \dotfill \\
    Escuela \dotfill \\
    Carrera o programa \dotfill \\
    Título al que opta \dotfill

    2. Autorización de Reproducción %seleccionar una opción

    % Este trabajo de titulación no puede reproducirse o transmitirse bajo ninguna forma o por ningún medio o procedimiento, sin permiso escrito del(os) autor(es), exceptuando la cita bibliográfica, resumen y metadatos que acreditan al trabajo y a su(s) autor(es).

    Se autoriza la reproducción total o parcial de este trabajo de titulación, con fines académicos, por cualquier medio o procedimiento, incluyendo la cita bibliográfica que acredita al trabajo y a su autor.
    
    En consideración a lo anterior, se autoriza su reproducción de forma (marque con una X):

    \begin{table}[h]
        \centering
        \setstretch{1}
        \renewcommand{\arraystretch}{2}
        \newcolumntype{Y}{>{\centering\arraybackslash}X}

        \begin{tabularx}{\textwidth}{|p{2cm}|X|}
            \hline
             & Inmediata \\
            \hline
             & A partir de la siguiente fecha: \rule{2cm}{0.4pt} (mes/año) \\
            \hline
        \end{tabularx}
    \end{table}

    \vfill

    Fecha: \rule{4cm}{0.4pt} \hfill Firma: \rule{4cm}{0.4pt} \hspace{2cm}

    \vspace{1cm}

    Esta autorización se otorga en el marco de la ley N°17.336 sobre Propiedad Intelectual, con carácter gratuito y no exclusivo para la Institución.
    
\end{singlespace}

    \hfill
\begin{minipage}{0.5\textwidth}
    \setstretch{1}
    \Large
        
    NOTA OBTENIDA: 

    \vspace{3cm}
    
    \rule{6.6cm}{0.4pt} \\
    Firma y timbre autoridad \\ responsable
\end{minipage}
\vfill


    % \chapter*{Dedicatoria}

    \begin{singlespace}
        \tableofcontents
        \listoftables
        \listoffigures
    \end{singlespace}

    % \textbf{RESUMEN:}

El resumen debe dar cuenta en forma clara y simple el contenido de la obra, permite determinar la pertinencia del trabajo y decidir al lector si el documento es de su interés. Se constituye de una descripción simple, breve y concisa de:

\begin{enumerate}
    \item El objetivo del trabajo
    \item Método o procedimiento utilizado
    \item Conclusiones o resultados obtenidos
\end{enumerate}

El resumen debe ser informativo y expresar en el mínimo número de palabras la mayor cantidad de información posible sobre el contenido del trabajo de titulación y su extensión máxima es de una página.

También incluye las palabras claves, que son términos temáticos que más destacan del trabajo, los cuales permiten su búsqueda e identificación en distintos recursos de información, como catálogos de biblioteca, motores de búsqueda y bases de datos.

\textbf{PALABRAS CLAVE:}

Repositorio, Universidad Tecnológica Metropolitana, software libre, preservación digital, acceso abierto.

\newpage

\textbf{ABSTRACT:}

LO MISMO QUE EL RESUMEN, PERO EN INGLÉS

In Chile, as in many other countries, the number of institutional repositories has grown steadily according to the need of information units and institutions in general to group, manage and preserve their institutional and scientific production, that is why this work addresses the elaboration of a best practices manual and general guidelines for the implementation of an institutional repository at Universidad Tecnológica Metropolitana, with emphasis in its value as a tool for supporting teaching and learning. It takes into account the situation of Chilean universities repositories and justify the need for implementing one at UTEM’s library system. Finally, it gives general guidelines for its implementation, which includes software, hardware, procedures and legal regulations.

\textbf{KEYWORDS:}

Repository, Universidad Tecnológica Metropolitana, open source software, digital preservation, open access.

    % ================ Cuerpo ==========================
    \mainmatter % números arábigos

    \chapter*{Introducción}
\addcontentsline{toc}{chapter}{Introducción}

\textit{Pentiente...}

% La introducción es la presentación clara, breve y precisa del contenido del trabajo y no debe incluir
% resultados ni conclusiones.
% 
% Comprende los siguientes aspectos:
% 
% a) Las razones que motivaron la elección del tema.
% 
% b) La justificación: los fundamentos que sustentan el tema y la relevancia del trabajo.
% 
% c) Una formulación clara del problema que se investigó, presentando los objetivos generales y la naturaleza del trabajo.
% 
% d) Un resumen de cómo se lograron los objetivos planteados.
% 
% e) Una orientación al lector de la forma en que se ha organizado el texto


    \chapter{Descripción de la Empresa y el Sistema}

\section{Descripción de la Empresa}

\subsection{Historia}

En 2005, el ingeniero Julio Gómez Vega comenzó a desarrollar un sistema de gestión de uniformes corporativos mientras trabajaba en la empresa textil La Scala. Tras el cierre de esta, fundó junto a Sebastián Gómez la empresa GyV Inversiones Ltda. en 2009, con el objetivo de ofrecer soluciones tecnológicas al rubro textil mediante la plataforma SISTAL.
Este sistema permitió automatizar la toma de medidas, pedidos y distribución de uniformes, consolidándose en empresas de gran tamaño como BancoEstado y Banco de Chile.
Durante más de una década, SISTAL fue una herramienta exitosa para la administración de uniformes, pero los avances tecnológicos y las nuevas exigencias en seguridad y escalabilidad motivaron su reingeniería hacia una arquitectura moderna basada en microservicios.

\subsection{Misión}

\begin{quote}
    Ser una empresa líder dentro del rubro textil, en lo que a sistemas de información se refiere, privilegiando la veracidad, rapidez y automatización de la información, con sistemas de bajo costo y fácil uso. Apostamos a la diferenciación en la comercialización mediante un servicio orientado al cliente final, con atención personalizada a cada usuario. Seremos considerados por nuestros clientes, más que un proveedor de servicios, un aliado estratégico que les ayudará a concentrarse específicamente en su negocio, dejando la administración de los uniformes corporativos en nuestras manos.
\end{quote}

\subsection{Visión}

\begin{quote}
    Formar parte de las mejores empresas de gestión de uniformes corporativos a nivel nacional e internacional, con modelos de vanguardia, una confección que cumpla con todos los estándares de calidad del mercado, y un servicio integral que exceda las expectativas de nuestros clientes. Para lograrlo, nuestro foco está apuntando a dos principios básicos, en primer lugar potenciar las fortalezas de nuestro recurso humano considerándolo como nuestro principal activo, y por supuesto, comprometernos en ir adoptando continuamente mejoras tecnológicas a nuestros sistemas informáticos.
\end{quote}

\subsection{Modelo de Negocios}

% - A que se dedica la empresa
% - principales fuentes de ingreso
% - principales catacteristicas
% - Lienzo canvas

Principales características del modelo de negocios:

\begin{itemize}
    \item Enfoque B2B (Business to Business): dirigido principalmente a empresas e instituciones que requieren gestionar la dotación de uniformes de su personal.
    \item Servicio integral: cubre todo el ciclo del proceso, desde el registro de tallas y pedidos hasta la confección y entrega de las prendas.
    \item Automatización del proceso: permite la gestión centralizada de información de usuarios, medidas y solicitudes mediante una plataforma web.
    \item Escalabilidad operativa: adaptable a distintas organizaciones, volúmenes de usuarios y flujos de trabajo.
% - Confirmar con el PO: 
    \item Modelo basado en licenciamiento y servicios: ingresos provenientes de la venta de licencias, soporte técnico y personalización del sistema.
\end{itemize}

En base al estudio realizado de la empresa, se diseñó el siguiente lienzo canvas (Figura \ref{fig:lienzo-canva-actual}).

\begin{figure}[htbp]
    \centering
    \includegraphics[width=\textwidth]{figuras/diagramas-actuales/lienzo-canva}
    \caption{Lienzo canva de la empresa actualmente}
    \label{fig:lienzo-canva-actual}
\end{figure}

\subsection{Áreas principales de la empresa}

% - descripción de las distintas áreas

\subsection{Estructura Organizacional}

% - Diagrama organizacional

\section{Descripción del Sistema (SISTAL)}

SISTAL es un sistema web de gestión de uniformes corporativos diseñado para atender a instituciones que requieren administrar de manera eficiente la dotación de prendas a su personal. La plataforma permite que cada funcionario registre sus medidas desde cualquier dispositivo con conexión a internet, información que el sistema procesa automáticamente para determinar las tallas correspondientes de cada prenda a fabricar. Posteriormente, la empresa encargada de la confección puede acceder y descargar los datos necesarios directamente desde el sistema, facilitando así la producción y el control de pedidos.

Este sistema consta de 3 actores principales en el proceso, los cuales tienen su propia vista y módulos específicos dentro del software:

\begin{itemize}
    \item \textbf{Funcionario}: Representa al usuario final perteneciente a la institución que utiliza los uniformes corporativos. Registra y actualiza sus medidas de tallaje, consulta el historial de pedidos y realiza solicitudes de cambio. El funcionario accede al sistema a través de una interfaz web desde cualquier dispositivo con conexión a internet.

    \item \textbf{Administrador (Institución)}: Actor encargado de la gestión y supervisión general del sistema. Tiene la capacidad de administrar usuarios, controlar los pedidos realizados por los funcionarios, aprobar solicitudes, generar reportes y mantener actualizada la información institucional. Además, el administrador es responsable de coordinar la comunicación entre la institución y el proveedor de uniformes, asegurando la correcta trazabilidad de los procesos y el cumplimiento de las políticas de dotación.

    \item \textbf{Proveedor (Fabricante de Uniformes)}: La empresa encargada de la confección y entrega de los uniformes. A través del sistema, el proveedor puede acceder a la información consolidada de tallas, cantidades y tipos de prendas requeridas, la cual puede descargar o consultar en línea para gestionar la producción. También puede actualizar el estado de los pedidos, registrar entregas y mantener comunicación con el administrador respecto a los plazos y disponibilidad de materiales.
\end{itemize}

\section{Objetivos}

\subsection{Objetivo General}

Diseñar la arquitectura completa del sistema de gestión de uniformes corporativos basado en microservicios en la nube, que sustituya al sistema actual y garantice mayor eficiencia operacional, escalabilidad y mantenibilidad.

% Se deberá poner esto aquí???
Como parte del proyecto, se desarrollará únicamente un módulo del sistema, a modo de implementación inicial y validación de la propuesta.


\subsection{Objetivos Específicos}

\begin{itemize}
    \item Analizar la arquitectura actual del sistema monolítico de gestión de uniformes, identificando limitaciones y oportunidades de mejora en escalabilidad y mantenibilidad.
    \item Analizar y documentar los flujos de datos a nivel de negocio con el propósito de modelarlos y replicarlos en la nueva arquitectura del sistema, asegurando su correcta adaptación y continuidad operativa en el proceso de migración.
    \item Diseñar una nueva arquitectura de microservicios escalable en la nube que permita la gestión eficiente del proceso de inventarios, pedidos, distribución y seguimiento de uniformes corporativos, considerando patrones de diseño modernos y mejores prácticas de desarrollo.
    \item Desarrollar un módulo del sistema, a modo de implementación inicial y validación de la propuesta.
\end{itemize}


\section{Alcances y Limitaciones}

\subsection{Alcances}

\begin{itemize}
    \item Reestructuración del sistema monolítico de SISTAL hacia una arquitectura de microservicios.
    \item Diseño de microservicios core para gestión de usuarios, inventarios, pedidos y estados.
    \item Configuración de infraestructura en la nube (GCP).
    \item Implementación de integración y despliegue continuo (CI/CD).
    \item Desarrollo e implementación de módulo de administración del sistema.
\end{itemize}

\subsection{Limitaciones}

\begin{itemize}
    \item No se contempla el desarrollo de aplicaciones móviles nativas o complementarias.
    \item No se usarán datos reales.
    \item No habrá capacitación masiva a usuarios finales.
    \item No se implementarán integraciones con sistemas externos no contemplados en la versión original.
    \item Se desarrollará únicamente el módulo de administración.
\end{itemize}



    \chapter{Marco Teórico}

\section{Área en estudio}

% En esta sección se describe el contexto general en el que se enmarca el proyecto, abordando el ámbito de los sistemas de gestión de uniformes corporativos y su relación con la gestión de recursos institucionales.
% Se puede incluir:
% 
% Breve descripción del dominio (gestión de uniformes, pedidos, tallajes, logística).
% 
% Importancia de la digitalización y automatización en la gestión de dotaciones.
% 
% Tendencias actuales en la gestión de inventarios y confección bajo demanda.

El presente trabajo se enmarca dentro del área de Ingeniería de Software, específicamente en el diseño y modernización de arquitecturas de sistemas de información.
El estudio aborda la reingeniería del sistema SISTAL, una plataforma web de gestión de uniformes corporativos, con el objetivo de transformar su estructura monolítica en una arquitectura basada en microservicios en la nube, utilizando prácticas y herramientas modernas de desarrollo, despliegue y mantenimiento de software.

Esta área contempla aspectos relacionados con la arquitectura de software, gestión de proyectos tecnológicos, seguridad de la información, DevOps e infraestructura en la nube, los cuales permiten optimizar la eficiencia operativa, escalabilidad y sostenibilidad técnica del sistema.

\section{Objetivos generales del área}

% Aquí se explican los propósitos del estudio teórico, como comprender las bases tecnológicas y metodológicas necesarias para rediseñar SISTAL.
% Ejemplo de redacción:
    % El objetivo del área teórica es analizar los fundamentos tecnológicos, arquitectónicos y metodológicos que permitan definir una propuesta de reestructuración del sistema SISTAL bajo un enfoque moderno, escalable y alineado con las necesidades del negocio.

El objetivo del área teórica es analizar los fundamentos tecnológicos, arquitectónicos y metodológicos que permitan definir una propuesta de reestructuración del sistema SISTAL alineado con las necesidades del negocio.
Este marco busca establecer las bases para el diseño de una arquitectura escalable, modular y flexible, apoyada en microservicios, computación en la nube y prácticas DevOps, que permita mejorar la eficiencia operativa, la mantenibilidad y la continuidad del servicio.

\section{Fundamentos Técnicos y Metodológicos del Proyecto}

La reestructuración del sistema SISTAL requiere comprender los fundamentos de los sistemas de información modernos, las arquitecturas de software orientadas a servicios y las metodologías que facilitan su desarrollo, despliegue y mantenimiento. A continuación, se describen los conceptos clave que orientan el diseño de la nueva solución.

... \textit{pendiente}

\begin{comment}
    \subsection{Sistemas de Información y su Rol en la Gestión Organizacional}

    Un \textit{sistema de información} es un conjunto integrado de componentes que permite capturar, procesar, almacenar y distribuir información para apoyar la toma de decisiones y el control dentro de una organización. Estos sistemas combinan personas, procesos, datos y tecnología, buscando mejorar la eficiencia operativa y la calidad del servicio.

    En el caso de la gestión de uniformes, un sistema de información como SISTAL facilita la trazabilidad de pedidos, la personalización de tallas y la coordinación entre los distintos actores involucrados (funcionarios, administradores y proveedores). Su adecuada arquitectura es esencial para garantizar la continuidad y confiabilidad de los procesos institucionales.

    \subsection{Arquitectura de Software}

    La \textit{arquitectura de software} define la estructura organizativa de un sistema y las relaciones entre sus componentes. Un diseño arquitectónico adecuado permite alcanzar cualidades como mantenibilidad, escalabilidad y flexibilidad ante cambios futuros.

    Tradicionalmente, muchos sistemas como SISTAL se desarrollaron bajo un \textit{modelo monolítico}, en el cual todas las funcionalidades están estrechamente integradas en una única aplicación. Si bien este enfoque simplifica el despliegue inicial, con el tiempo dificulta la incorporación de nuevas características, la escalabilidad independiente de módulos y la adopción de tecnologías modernas.

    \subsection{Arquitectura Basada en Microservicios}

    La \textit{arquitectura de microservicios} surge como una evolución del enfoque monolítico. Se caracteriza por descomponer el sistema en servicios pequeños, independientes y especializados, que se comunican entre sí mediante \textbf{interfaces bien definidas (APIs)}. Cada microservicio puede ser desarrollado, desplegado y escalado de forma autónoma, utilizando incluso diferentes tecnologías según sus requerimientos.

    Entre sus principales ventajas se encuentran:
    \begin{itemize}
        \item Mayor \textbf{escalabilidad} y facilidad para distribuir la carga de trabajo.
        \item Mejor \textbf{mantenibilidad}, al aislar los cambios en componentes específicos.
        \item Despliegues \textbf{más ágiles y seguros}, al no afectar todo el sistema.
    \end{itemize}

    Este enfoque se alinea plenamente con los objetivos del proyecto, ya que permite que SISTAL evolucione hacia una plataforma modular y adaptable a las demandas futuras.

    \subsection{Computación en la Nube}

    La \textit{computación en la nube} proporciona recursos de infraestructura, plataforma y software bajo demanda, lo que permite implementar soluciones escalables sin necesidad de invertir en hardware físico.

    Los tres modelos de servicio más comunes son:
    \begin{itemize}
        \item \textbf{IaaS (Infraestructura como Servicio):} Provisión de recursos básicos como máquinas virtuales y almacenamiento.
        \item \textbf{PaaS (Plataforma como Servicio):} Entornos de desarrollo y despliegue administrados.
        \item \textbf{SaaS (Software como Servicio):} Aplicaciones completas ofrecidas al usuario final a través de internet.
    \end{itemize}

    La adopción de una arquitectura basada en la nube permitirá que SISTAL reduzca costos de infraestructura, mejore la disponibilidad del servicio y optimice su capacidad de respuesta ante la demanda.

    \subsection{Contenedores y Orquestación}

    Los \textit{contenedores} son entornos ligeros que encapsulan una aplicación y todas sus dependencias, garantizando que se ejecute de manera uniforme en distintos entornos. Herramientas como \textbf{Docker} han revolucionado el despliegue de aplicaciones al facilitar la portabilidad y la replicación.

    La \textit{orquestación de contenedores}, mediante plataformas como \textbf{Kubernetes}, permite gestionar automáticamente el escalado, balanceo de carga, actualización y recuperación de servicios. Esto resulta esencial en arquitecturas de microservicios, donde la coordinación entre múltiples contenedores es crítica para mantener la estabilidad del sistema.

    \subsection{Metodologías DevOps}

    El enfoque \textbf{DevOps} promueve la integración entre los equipos de desarrollo (Dev) y operaciones (Ops) para lograr entregas continuas, automatización del ciclo de vida del software y una mejora constante del servicio.

    Entre sus principales prácticas se incluyen:
    \begin{itemize}
        \item \textbf{Integración continua (CI):} validación automática del código a medida que se desarrolla.
        \item \textbf{Entrega continua (CD):} despliegue rápido y controlado de nuevas versiones.
        \item \textbf{Monitoreo continuo:} supervisión proactiva del desempeño del sistema.
    \end{itemize}

    Aplicar DevOps en la reestructuración de SISTAL favorecerá una gestión más ágil de versiones, mayor estabilidad operativa y un mantenimiento preventivo del sistema.

    \subsection{Reingeniería de Software}

    La \textit{reingeniería de software} consiste en el análisis y rediseño de un sistema existente para mejorar su estructura interna, rendimiento y adaptabilidad sin alterar sus funciones esenciales. A diferencia de una migración total, la reingeniería busca \textbf{preservar el conocimiento del negocio}, optimizando la arquitectura y el código para cumplir con los estándares actuales de calidad y eficiencia.

    En el caso de SISTAL, la reingeniería implica rediseñar su arquitectura hacia un modelo distribuido en microservicios, manteniendo la lógica de negocio central y modernizando la infraestructura tecnológica.

    \subsection{Tecnologías y Herramientas Asociadas}

    El desarrollo del nuevo SISTAL se apoyará en tecnologías ampliamente utilizadas en entornos empresariales modernos, tales como:
    \begin{itemize}
        \item \textbf{Lenguajes y frameworks:} Java/Spring Boot, Node.js, React.
        \item \textbf{Bases de datos:} PostgreSQL, MongoDB.
        \item \textbf{Contenedores y orquestación:} Docker, Kubernetes.
        \item \textbf{Control de versiones y automatización:} Git, Jenkins, GitLab CI/CD.
        \item \textbf{Infraestructura en la nube:} AWS, Azure o Google Cloud.
    \end{itemize}

    Estas herramientas contribuirán a garantizar la estabilidad, escalabilidad y mantenibilidad del sistema rediseñado.

    \subsection{Síntesis del Marco Teórico}

    Los conceptos presentados conforman la base conceptual y tecnológica que orienta el rediseño del sistema SISTAL. La integración de microservicios, computación en la nube, contenedores y prácticas DevOps ofrece una plataforma sólida para construir un sistema más eficiente, escalable y alineado con las necesidades actuales y futuras de la organización.
\end{comment}


    \chapter{SISTAL}

En este capítulo se presenta un análisis detallado del sistema actual de gestión de uniformes corporativos, SISTAL, con el propósito de comprender su funcionamiento, estructura y alcance dentro de la organización. Se describen los principales procesos operativos, sus fortalezas, oportunidades, debilidades y amenazas mediante un análisis FODA, así como la propuesta de solución orientada a su reestructuración.
Asimismo, se incluye el diseño técnico del sistema vigente, representado a través de diagramas, los cuales permiten identificar los componentes críticos y las áreas susceptibles de mejora que fundamentan la propuesta de modernización del sistema.En este capítulo se profundizará sobre el sistema actual SISTAL, los problemas que enfrenta

\section{Análisis del Problema}

En esta sección se analizan las principales problemáticas del sistema Sistal (2008), tanto técnicas como funcionales. Se caracterizan las deficiencias que afectan la eficiencia, la escalabilidad y el mantenimiento, y se establecen los fundamentos para una mejora tecnológica.


\subsection{Situación Actual}

SISTAL es una aplicación web monolítica desarrollada en PHP (2008), con base de datos MySQL sin relaciones, alojada en hosting compartido (BlueHosting) y sin control de versiones. Atiende tres actores: Administrador (empresa contratante), Funcionario (empleado) y Proveedor (confección/entrega).
Funcionalmente permite: administración de funcionarios y segmentos, captura de tallas, consolidación y envío de órdenes de confección al proveedor, seguimiento básico de estados (ingresado, confección, despacho y entrega), registro de entregas e incidencias y listados/reportes simples.

\subsubsection{Hallazgos técnicos y operativos}


\begin{itemize}
    \item \textbf{Arquitectura y datos}: monolito acoplado; modelo de datos sin normalización, lo que limita integridad y reporting.
    \item \textbf{Procesos de desarrollo}: sin control de versiones ni despliegue automatizado, sin trazabilidad de cambios ni ambientes reproducibles.
    \item \textbf{Infraestructura}: hosting compartido con latencias e inestabilidad; sin observabilidad ni alta disponibilidad.
    \item \textbf{Seguridad}: controles de acceso y cifrado insuficientes, sin prácticas DevSecOps.
    \item \textbf{UX/UI}: interfaz obsoleta y no responsive; fricción en la captura de información.
    \item \textbf{Escalabilidad}: inexistencia de multi-tenant; dificultades para servir a múltiples empresas y para integrar proveedores/logística/ERP.
\end{itemize}

\subsection{Análisis FODA}

\subsection{Solución Propuesta}

\section{Diseño del Sistema Actual}

En esta sección se documenta el diseño técnico del sistema actual, representando su estructura interna y sus principales componentes. A través de diagramas de casos de uso, base de datos, flujo, arquitectura y despliegue, se busca comprender la organización del sistema existente y las áreas que requieren modernización.


\subsection{Diagrama de Casos de Uso}

\textit{Falta ordenar el diagrama de casos de uso, por el momento se encuentra en el siguiente enlace:} \url{https://excalidraw.com/#room=80d12a9b923fafaeae02,PoQkDRXp5qahft2VZWr_gw}

\subsection{Diagrama de Bases de Datos}

Los diagramas de base de datos presentados en las Figuras \ref{fig:diagrama-bdd-1-actual} y \ref{fig:diagrama-bdd-2-actual}, son las tablas que se encuentran actualmente en el sistema. Como se aprecia, no contienen relaciones y muchas de las columnas no se utilizan.

\begin{figure}[htbp]
    \centering
    \includegraphics[height=0.9\textheight]{figuras/diagramas-actuales/diagrama-bdd-1}
    \caption{Diagrama físico de bases de datos del sistema actual (Parte 1)}
    \label{fig:diagrama-bdd-1-actual}
\end{figure}

\begin{figure}[htbp]
    \centering
    \includegraphics[height=0.9\textheight]{figuras/diagramas-actuales/diagrama-bdd-2}
    \caption{Diagrama físico de bases de datos del sistema actual (Parte 2)}
    \label{fig:diagrama-bdd-2-actual}
\end{figure}


\subsection{Diagrama de Flujo}

\textit{Falta añadir descripciones para los 3 diagramas de Flujo:} Figuras \ref{fig:diagrama-flujo-actual-registro-de-tallas}, \ref{fig:diagrama-flujo-actual-envio-a-confeccion} y \ref{fig:diagrama-flujo-actual-post-venta}.

\begin{figure}[htbp]
    \centering
    \includegraphics[height=0.9\textheight]{figuras/diagramas-actuales/diagrama-flujo-actual-registro-de-tallas.png}
    \caption{Diagrama de flujo del sistema actual sobre el proceso de registro de tallas}
    \label{fig:diagrama-flujo-actual-registro-de-tallas}
\end{figure}

\begin{figure}[htbp]
    \centering
    \includegraphics[height=0.9\textheight]{figuras/diagramas-actuales/diagrama-flujo-actual-envio-a-confeccion.png}
    \caption{Diagrama de flujo del sistema actual sobre el proceso de envío a confección}
    \label{fig:diagrama-flujo-actual-envio-a-confeccion}
\end{figure}

\begin{figure}[htbp]
    \centering
    \includegraphics[height=0.9\textheight]{figuras/diagramas-actuales/diagrama-flujo-actual-post-venta.png}
    \caption{Diagrama de flujo del sistema actual sobre el proceso de post venta}
    \label{fig:diagrama-flujo-actual-post-venta}
\end{figure}

\subsection{Diagrama de Arquitectura}

En el diagrama de arquitectura actual (Figura \ref{fig:diagrama-arq-actual}) se aprecia la configuración monolítica del sistema, como este se encuentra alojado en BlueHosting que contienen ambos, el sistema y la base de datos. Con los actores principales conectándose a este mismo.

\begin{figure}[htbp]
    \centering
    \includegraphics[width=\textwidth]{figuras/diagramas-actuales/diagrama-de-arquitectura}
    \caption{Diagrama de arquitectura del sistema actual}
    \label{fig:diagrama-arq-actual}
\end{figure}

\subsection{Diagrama de Despliegue}

En el diagrama de despliegue actual (Figura \ref{fig:diagrama-despliegue-actual})) se muestra el proceso manual de despliegue y actualización del sistema, en el que desde un entorno local, a través del protocolo \code{FTP/SFTP} se realiza una conexión al proveedor BlueHosting y se añaden/reemplazan el código del sistema.

\begin{figure}[htbp]
    \centering
    \includegraphics[height=0.9\textheight]{figuras/diagramas-actuales/diagrama-de-despliegue}
    \caption{Diagrama de despliegue del sistema actual}
    \label{fig:diagrama-despliegue-actual}
\end{figure}



    \chapter*{Conclusiones}
\addcontentsline{toc}{chapter}{Conclusiones}

\textit{Pendiente...}

% Las conclusiones pueden incluir los resultados obtenidos en la investigación, comprobación o refutación de la hipótesis, recomendaciones que puedan ser útiles al problema de investigación, reflejando a su vez los alcances y limitaciones del trabajo, aportes al campo o disciplina del conocimiento y conclusiones generales.
% 
% Deben tener una redacción clara, concreta y directa; no son un resumen de la investigación.


    % \chapter{Cheatsheet}

% Formato de texto
\textbf{Negrita}, \textit{Itálica}, \underline{Subrayado}  

% Listas
\begin{itemize}
    \item Primer ítem
    \item Segundo ítem
\end{itemize}

\begin{enumerate}
    \item Ítem 1
    \item Ítem 2
\end{enumerate}

% Ecuaciones
Ecuación en la misma línea: $a^2 + b^2 = c^2$

Ecuación en medio: $$E = mc^2$$

% Figuras
\begin{figure}[htbp]
    \centering
    \includegraphics[width=0.3\linewidth]{figuras/frog.jpg}
    \caption{Ejemplo de cómo presentar una ilustración utilizando una foto del monumento a la brigada del Negev}
    \label{fig:ejemplo}
    \sourcefig{Israel Tour Guides}{2013}{https://israel-tourguide.info/2013/04/20/photo-of-the-week-negev-brigade-monument}
\end{figure}

% Tablas
\begin{table}[htbp]
    \centering
    \caption{Tabla de ejemplo}
    \begin{tabular}{|c|c|c|}
    \hline
    A & B & C \\ \hline
    1 & 2 & 3 \\ \hline
    4 & 5 & 6 \\ \hline
    \end{tabular}
    \label{tab:ejemplo}
\end{table}

% Referencias cruzadas
Ver Figura \ref{fig:ejemplo} y Tabla \ref{tab:ejemplo}.

% Citas y bibliografía
Ejemplo de cita: \cite{dinosaurios2006}; \parencite{dinosaurios2006}

% Enlaces
Ejemplo de enlace: \href{https://www.overleaf.com}{Overleaf}

Ejemplo de url: \url{https://www.overleaf.com}

% Código
\begin{lstlisting}[language=Python, caption=Hola Mundo en Python]
def saludo():
    print("Hola, mundo!")

saludo()
\end{lstlisting}

\begin{sloppypar}
    Palabras monoespaciadas: El directorio \code{\%HOME} se encuentra en \code{/home/<usuario>}
\end{sloppypar}

    % ================ Bibliografía ====================

    % \printbibliography[heading=bibintoc]

    % ================ Apendice ========================
    % Apendice
    % \include{apendice}

\end{document}
