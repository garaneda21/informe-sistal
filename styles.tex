\definecolor{light-gray}{gray}{0.95}

\titlespacing*{\chapter}{0pt}{0pt}{20pt}  % centrado, sin espacio arriba
% \titlespacing*{name=\chapter,numberless}{0pt}{0pt}{20pt}  % centrado, sin espacio arriba

\titleformat{\chapter}[display] % Estilo de los capítulos
  {\bfseries\Large\centering}
  {\MakeUppercase\chaptername\ \thechapter}
  {10pt}
  {\titlerule\MakeUppercase}

\titleformat{name=\chapter,numberless}[block]
  {\bfseries\Large\centering}        % mismo estilo
  {}                                % sin número
  {0pt}                             % separación antes de la línea
  {\MakeUppercase}                  % título en mayúsculas
  [\vspace{10pt}\titlerule]         % línea después del título
\titlespacing*{name=\chapter,numberless}{0pt}{0pt}{20pt}

\titleformat{\section}[block]
  {\Large\bfseries}
  {\thesection}
  {10pt}
  {\MakeUppercase}

\titleformat{\subsection}[block]
  {\large\bfseries}
  {\thesubsection}
  {10pt}
  {\MakeUppercase}

\titleformat{\subsubsection}[block]
  {\normalsize\bfseries}
  {\thesubsubsection}
  {10pt}
  {\MakeUppercase}

\etocsettocstyle{\chapter*{Índice General}}{}

% 'Capítulo X: ' en Índice
\renewcommand{\cftchappresnum}{Capítulo~}
\renewcommand{\cftchapaftersnum}{:\ }
\renewcommand{\cftchapnumwidth}{6em}

% Colores de Enlaces (Links)
\hypersetup{
    colorlinks,
    citecolor=black,
    filecolor=black,
    linkcolor=black,
    urlcolor=blue
}

% Configuración del estilo de código
\lstset{
    basicstyle=\ttfamily\footnotesize,
    keywordstyle=\color{blue}\bfseries,
    commentstyle=\color{gray}\itshape,
    stringstyle=\color{teal},
    backgroundcolor=\color{light-gray},
    numbers=left,
    numberstyle=\scriptsize,
    stepnumber=1,
    numbersep=8pt,
    breaklines=true,
    tabsize=4,
    showstringspaces=false
    aboveskip=5pt,
    belowskip=5pt,
    lineskip=-1pt
}

% Modificar Headers y Footers
\pagestyle{fancy}
\fancyhf{}
\renewcommand{\headrulewidth}{0pt} % quitar línea superior
\fancyfoot[R]{\thepage}
\fancypagestyle{plain}{}

%%% COMANDOS PERSONALIZADOS %%% 

% \sourcefig{nombre}{año}{url} : para insertar una fuente a una figura
\newcommand{\sourcefig}[3]{%
  \caption*{%
    \small\textcolor{gray}{%
      Fuente: #1%
      \if\relax\detokenize{#2}\relax\else, #2\fi%
      \if\relax\detokenize{#3}\relax%
        % No hay URL → no mostrar "Enlace:"
      \else%
        , Enlace: \url{#3}%
      \fi%
    }%
  }%
}
% \code{} : para insertar una palabra de texto monoespaciada, como documentación
\newcommand{\code}[1]{\colorbox{light-gray}{\texttt{#1}}}
