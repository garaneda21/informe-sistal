\textbf{RESUMEN:}

El resumen debe dar cuenta en forma clara y simple el contenido de la obra, permite determinar la pertinencia del trabajo y decidir al lector si el documento es de su interés. Se constituye de una descripción simple, breve y concisa de:

\begin{enumerate}
    \item El objetivo del trabajo
    \item Método o procedimiento utilizado
    \item Conclusiones o resultados obtenidos
\end{enumerate}

El resumen debe ser informativo y expresar en el mínimo número de palabras la mayor cantidad de información posible sobre el contenido del trabajo de titulación y su extensión máxima es de una página.

También incluye las palabras claves, que son términos temáticos que más destacan del trabajo, los cuales permiten su búsqueda e identificación en distintos recursos de información, como catálogos de biblioteca, motores de búsqueda y bases de datos.

\textbf{PALABRAS CLAVE:}

Repositorio, Universidad Tecnológica Metropolitana, software libre, preservación digital, acceso abierto.

\newpage

\textbf{ABSTRACT:}

LO MISMO QUE EL RESUMEN, PERO EN INGLÉS

In Chile, as in many other countries, the number of institutional repositories has grown steadily according to the need of information units and institutions in general to group, manage and preserve their institutional and scientific production, that is why this work addresses the elaboration of a best practices manual and general guidelines for the implementation of an institutional repository at Universidad Tecnológica Metropolitana, with emphasis in its value as a tool for supporting teaching and learning. It takes into account the situation of Chilean universities repositories and justify the need for implementing one at UTEM’s library system. Finally, it gives general guidelines for its implementation, which includes software, hardware, procedures and legal regulations.

\textbf{KEYWORDS:}

Repository, Universidad Tecnológica Metropolitana, open source software, digital preservation, open access.